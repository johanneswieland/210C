\documentclass[english,xcolor=svgnames]{beamer}

\input{../../../../Templates/Latex/teachingslidesbeamer.tex}


% ===========================================================
% ===========================================================
% ===========================================================
\begin{document}

\title{Monopolistic Competition and Sticky Prices}
\vspace{1cm}
\author[shortname]{
\begin{tabular}{c}
	Johannes Wieland \\ 
	\footnotesize \href{mailto:jfwieland@ucsd.edu}{jfwieland@ucsd.edu}  \\ 
\end{tabular}
}

\date{Spring \the\year}

\setbeamertemplate{footline}{}
\makebeamertitle
\setbeamertemplate{footline}[frame number]{}

\addtocounter{framenumber}{-1}

%%%%%%%%%%%%%%%%%%%%%%%%%%%%%%%%%%%%%%%%%%%%%%%%%%
\AtBeginSection[]{
\setbeamertemplate{footline}{}
  \frame<beamer>{ 

    \frametitle{Outline}   

    \tableofcontents[currentsection] 
  }
\setbeamertemplate{footline}[frame number]{}
\addtocounter{framenumber}{-1}
}

\AtBeginSubsection[]{
\setbeamertemplate{footline}{}
  \frame<beamer>{ 

    \frametitle{Outline}   

    \tableofcontents[currentsection,currentsubsection] 
  }
  \setbeamertemplate{footline}[frame number]{}
  \addtocounter{framenumber}{-1}
}


\section{Introduction}



\begin{frame}
\frametitle{Monopolistic Competition and Markups}
\begin{itemize}
	\item Goal: Add nominal rigidity for non-neutrality.
	\item Problem: How does nominal rigidity work with CRS and
perfect competition?
	\begin{itemize}
		\item Older literature: Rationing with output determined as minimum of supply and demand at given price.
		\item Newer Literature: Get rid of CRS and perfect competition and replace with IRS and imperfect competition $\Rightarrow$ firms set prices.
	\end{itemize}
	\item But how do we have features of oligopoly without modeling the industrial organization, which is a mess in GE?
	\item Blanchard and Kiyotaki (1987) and subsequent literature: Use monopolistic competition.
	\begin{itemize}
		\item Idea going back to Chamberlain (1933), but popularized by tractable setup of Dixit and Stiglitz (1977).
	\item Monopolistic competition is widely used in GE modeling (macro, trade, labor, etc.) and is something you should know.
	\end{itemize}
\end{itemize}
\end{frame}

%\begin{frame}
%\frametitle{Monopolistic Competition and Markups}
%\begin{enumerate}[1.]
%	\item Dixit-Stiglitz Preferences and Production
%	\item Markups and Monopolistic Competition
%	\item RBC With Monopolistic Competition: The Frictionless Benchmark
%	\item One Period Nominal Rigidity
%\end{enumerate}
%\end{frame}


%%%%%%%%%%%%%%%%%%%%%%%%%%%%%%%%%%%%%%%%%%%%%%%%%%
\section{Dixit-Stiglitz}
%%%%%%%%%%%%%%%%%%%%%%%%%%%%%%%%%%%%%%%%%%%%%%%%%%

\begin{frame}
\frametitle{Monopolistic Competition}
\begin{itemize}
	\item Continuum of goods (``varieties'') $i\in [0, 1]$ with a monopolist for each good.
	\item Each monopolist faces a downward-sloping demand curve.
	\begin{itemize}
		\item Substitution between goods imperfect due to ``love of variety.''
	\end{itemize}
	\item Each monopolist's optimal choice has an infinitesimal effect on economy-wide aggregates.
	\begin{itemize}
		\item Industrial organization in GE is simple.
		\item Imperfect competition without game theory.
	\end{itemize}
	\item Today: demand curve from consumer preferences.
	\item Can equivalently do firm optimization problem.
\end{itemize}
\end{frame}



\begin{frame}
\frametitle{Household Problem: Setup and Notation}
\begin{itemize}
\item Idea: CES over a continuum of goods:
\begin{align*}
	E_t \left\{\sum_{s=0}^{\infty}\beta^s\left(\frac{C_{t+s}^{1-\gamma}}{1-\gamma}+\zeta\frac{(M_{t+s}/P_{t+s})^{1-\nu}}{1-\nu}-\chi \frac{N_{t+s}^{1+\varphi}}{1+\varphi}\right)\right\}
\end{align*}
where
\begin{align*}C_t=\left[\int_0^1C_t(i)^{\frac{\epsilon-1}{\epsilon}}di\right]^{\frac{\epsilon}{\epsilon-1}}\text{ with }\epsilon>0
\end{align*}
	\item Budget constraint:
\begin{align*}
		\int_0^1 P_t(i) C_{t}(i)di+B_t+M_t&\le Q_{t-1}B_{t-1}+M_{t-1}+W_t N_t \\& + P_{t}(TR_t+PR_t)
	\end{align*}	
	\item Ct is sometimes called a ``Dixit-Stiglitz aggregate.''
\end{itemize}
\end{frame}


\begin{frame}
\frametitle{Solving Dixit-Stiglitz: Two-Stage Budgeting}
\begin{itemize}
	\item Two-Stage Budgeting Theorem (Deaton and Muellbauer):
\begin{itemize}
	\item If upper stage is separable and lower stage is homothetic, can use two-stage budgeting with nested preferences.
	\item[$\Rightarrow$] Solve the inner nest taking expenditure as given and outer nest by standard utility maximization given inner nest optimization to determine expenditure on bundle purchased in inner nest.
	\end{itemize}
%	\item Example:
%	\begin{align*}
%		U=C_t^\mu H_t^{1-\mu}\text{ where }C_t=\left[\int_0^1 C_t(i)^{\frac{\epsilon-1}{\epsilon}}di\right]^{\frac{\epsilon}{\epsilon-1}}
%	\end{align*}
%	\begin{itemize}
%	\item CES is homothetic and C-D is separable (after taking logs).
%	\item Cost minimize for $C_t(i)$ as a function of $C_t$ and then use C-D:
%	\begin{align*}
%		\mu=C_tP_{C,t}/Y_t\text{ and }1-\mu=H_tP_{H,t}/Y_t
%	\end{align*}
%	\end{itemize}
	\item We can use two-stage budgeting here.
\end{itemize}
\end{frame}


\begin{frame}
\frametitle{Dixit-Stiglitz: Inner Nest Maximization}
\begin{align*}
	\max \left[\int_0^1 C_t(i)^{\frac{\epsilon-1}{\epsilon}}di\right]^{\frac{\epsilon}{\epsilon-1}}-\lambda\left(\int_0^1 P_t(i) C_{t}(i)di-X_t\right)
\end{align*}
\begin{itemize}
	\item Xt is expenditure on Dixit-Stiglitz goods.
	\item FOC
	\begin{align*}
		C_t(i)^{-\frac{1}{\epsilon}}C_t^{\frac{1}{\epsilon}}=\lambda P_t(i)
	\end{align*}
	\item For any two goods i and j,
	\begin{align*}
		C_t(i)=C_t(j)\left(\frac{P_t(i)}{P_t(j)}\right)^{-\epsilon}=\frac{P_t(i)^{-\epsilon}}{P_t(j)^{1-\epsilon}}P_{t}(j)C_{t}(j)
	\end{align*}
	\item Bring the denominator over and integrate wrt j:
	\begin{align*}
		C_t(i)\int_0^1 P_t(j)^{1-\epsilon}dj&=P_t(i)^{-\epsilon}\int_0^1 P_{t}(j)C_{t}(j)dj \\
		C_t(i)&=\frac{P_t(i)^{-\epsilon}}{\int_0^1 P_t(j)^{1-\epsilon}dj}X_t
	\end{align*}
\end{itemize}
\end{frame}


\begin{frame}
\frametitle{Dixit-Stiglitz: Price Index}
\begin{itemize}
	\item Indirect utility is:
	\begin{align*}
		v(P_t(k)|_{k=0}^{1},X_t) &= \left[\int_0^1 C_t(i)^{\frac{\epsilon-1}{\epsilon}}di\right]^{\frac{\epsilon}{\epsilon-1}}\\
		&=\frac{X_t}{\left[\int_0^1 P_t(j)^{1-\epsilon}dj\right]^{\frac{1}{1-\epsilon}}}
	\end{align*}
	\item The cost of buying one unit of utility (=one unit of consumption) is:
	\begin{align*}
		P_t = \left[\int_0^1 P_t(j)^{1-\epsilon}dj\right]^{\frac{1}{1-\epsilon}}
	\end{align*}
	\begin{itemize}
%			\item This is an ideal price index.
			\item Index is geometric weighted average of individual good prices.
		\end{itemize}
\end{itemize}
\end{frame}

\begin{frame}
\frametitle{Dixit-Stiglitz: Demand Function}
\begin{itemize}
	\item $X_t=P_tC_t$, so plugging in price index gives
	\begin{align*}C_t(i)&=\frac{P_t(i)^{-\epsilon}}{ P_t^{1-\epsilon}}X_t \\
	&=\left(\frac{P_t(i)}{ P_t}\right)^{-\epsilon} C_t
	\end{align*}
	\item  CES structure delivers \emph{constant elasticity demand function}.
	\begin{itemize}
		\item Elasticity of demand is elasticity of substitution $\epsilon$.
		\item As $\epsilon\rightarrow\infty$, perfect substitutes and demand perfectly elastic.
		\item As $\epsilon\rightarrow 1$, less perfect substitutes and demand more inelastic (but still elastic as $\epsilon>1$).
	\end{itemize}
	\item Each firm has infinitesimal impact on $C_t$ and $P_t$ and treats them as exogenous.
\end{itemize}
\end{frame}


\begin{frame}
\frametitle{Solving Dixit-Stiglitz: Upper Stage
}
\begin{itemize}
	\item With this budget constraint can be written as:
	\begin{align*}
		P_t C_{t}+B_t+M_t&\le Q_{t-1}B_{t-1}+M_{t-1}+W_t N_t + P_{t}(TR_t+PR_t)
	\end{align*}
	\item Solve upper stage as normal with $C_t$ as Dixit-Stiglitz aggregate:
	\begin{align*}
	\frac{W_t}{P_t}&=\frac{\chi N_t^\varphi}{C_t^{-\gamma}} \\
		1&=\beta E_t\left\{Q_t \frac{P_t}{P_{t+1}} \frac{C_{t+1}^{-\gamma}}{C_{t}^{-\gamma}}\right\}=E_t\{\Lambda_{t,t+1}R_{t+1}\} \\
		1&=\beta E_t\left\{\frac{P_t}{P_{t+1}} \frac{C_{t+1}^{-\gamma}}{C_{t}^{-\gamma}}\right\}+\zeta\frac{(M_t/P_t)^{-\nu}}{C_{t}^{-\gamma}}
	\end{align*}
	\end{itemize}
\end{frame}


\begin{frame}
\frametitle{Uses of Dixit-Stiglitz
}
\begin{itemize}
	\item Dixit-Stiglitz is frequently used in GE modeling both in macro and other subfields.
	\begin{itemize}
		\item Often along with free entry margin that drives profits to zero and endogenously determines number of products.
	\end{itemize}
	\item Noteworthy Examples:
	\begin{itemize}
		\item ``New'' Trade Theory (Krugman, 1980): Love of variety explains high volume of intra-industry trade, e.g. Japan exports Lexus to Germany and Germany exports Mercedes to Japan.
		\item New Economic Geography (Krugman, 1990): Urbanization determined by balance between dispersion forces (e.g., housing supply) and agglomeration forces created by increasing returns. As trade costs fall, cities should develop.
		\item Endogenous Growth Theory (Romer, 1990): Profits give entrepreneurs incentives to invest in creating new products. Growth through endogenously expanding product variety.
	\end{itemize}
\end{itemize}
\end{frame}


\begin{frame}
\frametitle{Dixit-Stiglitz Production
}
\begin{itemize}
	\item Dixit-Stiglitz is used two ways:
\begin{itemize}
	\item Preferences: Households consume each good $i$,
CES preferences over continuum of goods.
	\item Production: Households consume final good assembled from
intermediates $i$, CES production fn over continuum of goods.
\end{itemize}
	\item These are essentially equivalent.
	\begin{itemize}
	\item We used utility maximization given to expenditure $X_t$, but same as cost minimization (duality theory).
	\item Cost min s.t. D-S utility level $C_t$ mathematically equivalent to profit max s.t. CES production is $C_t$ (up to sign change).
	\item Intuition: Does not matter where continuum is as long as it as CES structure.
\end{itemize}
\item Gali book presents model using Dixit-Stiglitz preferences. I will follow this convention here.
%\item I will use Dixit-Stiglitz production.
\end{itemize}
\end{frame}



%%%%%%%%%%%%%%%%%%%%%%%%%%%%%%%%%%%%%%%%%%%%%%%%%%
\section{Monopolistic Competition and Sticky Prices}
%%%%%%%%%%%%%%%%%%%%%%%%%%%%%%%%%%%%%%%%%%%%%%%%%%

\begin{frame}
\frametitle{Intermediate Good Producers}
\begin{itemize}
	\item From the household problem, each monopolist faces a downward-sloping demand curve:
	\begin{align*}
		C_t(i)&=\left(\frac{P_t(i)}{ P_t}\right)^{-\epsilon} C_t
	\end{align*} 
	\item Produce variety CRS with labor:
	\begin{align*}
		Y_t(i)=A_t N_t(i)
	\end{align*}
	\begin{itemize}
		\item See Gali for DRS.
	\end{itemize}
	\item Market clearing implies $C_t(i)=Y_t(i)$ and $C_t=Y_t$.
\end{itemize}
\end{frame}

\begin{frame}
\frametitle{Profit Maximization with Flexible Prices}
\begin{itemize}
	\item Profits for the monopolist are
	\begin{align*}
		PR_t(i)&=\frac{P_t(i)}{P_t}Y_t(i) - \frac{W_t}{P_t}N_t(i)  \\
		&= \left(\frac{P_t(i)}{ P_t}\right)^{1-\epsilon}Y_t -\frac{W_t}{P_t}  \left(\frac{P_t(i)}{ P_t}\right)^{-\epsilon}  \frac{Y_t}{A_t}
	\end{align*}
	\item FOC for $P_t(i)$:
	\begin{align*}
		(1-\epsilon)\left(\frac{P_t^*(i)}{ P_t}\right)^{-\epsilon}Y_t + \epsilon\frac{W_t}{P_t}  \left(\frac{P_t^*(i)}{ P_t}\right)^{-\epsilon-1}  \frac{Y_t}{A_t} = 0
	\end{align*}
	\item The optimal real price is:
	\begin{align*}
		\frac{P_t^*(i)}{ P_t} = \underbrace{\mu}_{\text{Mark-up}} \times \underbrace{\frac{W_t}{P_t}\frac{1}{A_t}}_{\text{Real MC}},\qquad \mu = \frac{\epsilon}{\epsilon-1}
	\end{align*}
	\begin{itemize}
		\item Real price is a multiplicative markup over real marginal cost.
	\end{itemize}
%	\item Real price is a multiplicative markup over marginal cost.
%	\begin{itemize}
%		\item Markup is inversely related to elasticity of demand.
%		\item Monopolist always on elastic portion of demand curve.
%	\end{itemize}
\end{itemize}
\end{frame}

\begin{frame}
\frametitle{Classical Dichotomy}
\begin{itemize}
	\item Monopolistic competition by itself does not deliver monetary non-neutrality.
	\item Labor demand equation becomes
	\begin{align*}
		 \frac{W_t}{P_t} = \underbrace{\frac{1}{\mu}}_{\text{Inverse of Mark-up}} \times \underbrace{A_t}_{MPL_t}
	\end{align*}
	\item This is the only change $\Rightarrow$ Classical Dichotomy holds.
	\item To get non-neutrality of money need nominal rigidity.
	\begin{itemize}
		\item Monopolistic competition gives an identity to the price setter. We can study what happens when firms keep prices fixed.
	\end{itemize}
\end{itemize}
\end{frame}

\begin{frame}
\frametitle{Intermediate Good Producers: Calvo Assumption}
\begin{itemize}
	\item Calvo (1983) pricing assumption: Each firm resets price each period with iid probability $1 - \theta$.
	\begin{itemize}
		\item By LLN, fraction that reset is $1 - \theta$ and fraction constant is $\theta$.
		\item Average price duration follows geometric dist with mean
duration $\frac{1}{1-\theta}$.
	\end{itemize}
	\item Firms that adjust prices choose $P_t(i), Y_t(i), N_t(i)$ to maximize expected discounted profits and demand.
	\item Firms that do not adjust prices set output to meet demand.
%	 as long as Pt (i) > MCtn (i) (nominal MC).
\end{itemize}
\end{frame}


\begin{frame}
\frametitle{Intermediate Good Producers: Calvo Assumption}
\begin{itemize}
	\item Calvo is a strong assumption!
	\item Is the world Calvo?
	\begin{itemize}
			\item Literally, no.
			\item But it could be a decent approximation.
	\end{itemize}
	\item Literature on ``menu cost'' models where there is an inaction region due to fixed cost of changing price.
	\begin{itemize}
		\item Initial literature: Much more flexible than Calvo, since firms that have price furthest from MC change price.
		\item Recent literature: To match micro-pricing facts, need large and infrequent firm-level MC shocks, which looks close to Calvo.
	\end{itemize}
\end{itemize}
\end{frame}

\begin{frame}
\frametitle{Price Dynamics With Calvo}
\begin{itemize}
	\item A fraction $1-\theta$ of firms adjust to $P_t^*$ and fraction $\theta$ keep $P_{t-1}(i)$:
	\begin{align*}P_t&=\left[\int_0^1 P_t(i)^{1-\epsilon}di\right]^{\frac{1}{1-\epsilon}}\\
		&=\left[\theta \int_0^1 P_{t-1}(i)^{1-\epsilon}di + (1-\theta)\int_0^1 P_{t}^{*1-\epsilon}di\right]^{\frac{1}{1-\epsilon}} \\
		&=\left[\theta \left(\int_0^1 P_{t-1}(i)^{1-\epsilon}di\right)^{\frac{1-\epsilon}{1-\epsilon}} + (1-\theta)\int_0^1 P_{t}^{*1-\epsilon}di\right]^{\frac{1}{1-\epsilon}}\\
		&=\left[\theta P_{t-1}^{1-\epsilon} + (1-\theta) P_{t}^{*1-\epsilon}\right]^{\frac{1}{1-\epsilon}}
	\end{align*}
	\item Price index $P_t$ is geometric average of $P_{t-1}$ and $P_t^*$.
	\item Calvo is tractable because we do not need to keep track of distribution of prices.
\end{itemize}
\end{frame}


\begin{frame}
\frametitle{Inflation Dynamics With Calvo
}
\begin{itemize}
	\item Divide by $P_{t-1}$ to get inflation between $t-1$ and $t$, $\Pi_t$ 
	\begin{align*}\Pi_t =\frac{P_t}{P_{t-1}}=\left[\theta  + (1-\theta) \left(\frac{P_{t}^{*}}{P_{t-1}}\right)^{1-\epsilon}\right]^{\frac{1}{1-\epsilon}}
	\end{align*}
	\item From this, we can see that Calvo price setting implies a partial adjustment mechanism:
	\begin{itemize}
		\item If $P_t^*=P_{t-1}$, then $\Pi_t=1$.
		\item If $P_t^*>P_{t-1}$, then $\Pi_t>1$ and $P_t^*>P_t> P_{t-1}$.
	\end{itemize}
	\end{itemize}
\end{frame}


\begin{frame}
\frametitle{Optimal Intermediate Reset Price Setting
}
\begin{align*}
	\max_{P_{t}^*,\{Y_{t+s|t}\}_{s=0}^{\infty}}E_t &\left\{\sum_{s=0}^{\infty}\theta^s\Lambda_{t,t+s}\left(\frac{P_t^*Y_{t+s|t}}{P_{t+s}} -TC_{t+s}(Y_{t+s|t})\right)\right\}\\
	&Y_{t+s|t}=\left(\frac{P_t^*}{P_{t+s}}\right)^{-\epsilon} Y_{t+s}
\end{align*}
\begin{itemize}
	\item The intermediate producer maximizes real discounted profits.
	\begin{itemize}
		\item Also discounting by prob they keep price same $\theta$.
	\end{itemize}
	\item Real Profits are:
	\begin{itemize}
		\item Nominal revenue  deflated by the price level, $\frac{P_t^*Y_{t+s|t}}{P_{t+s}}$, minus total real cost $TC_{t+s}(Y_{t+s|t})$.
		\item Output at time $t+s$, $Y_{t+s|t}$ is determined by the demand curve at time $t+s$ and the price chosen at time $t$.
	\end{itemize}
\end{itemize}
\end{frame}


\begin{frame}
\frametitle{Optimal Intermediate Reset Price Setting
}
\begin{align*}
	E_t \left\{\sum_{s=0}^{\infty}\theta^s\Lambda_{t,t+s}Y_{t+s|t}\left(\frac{P_t^*}{P_{t+s}} - (1+\mu) MC_{t+s}(Y_{t+s|t})\right)\right\}&=0
\end{align*}
\begin{itemize}
	\item If $\theta=0$, no stickiness and this collapses to flex price model:
	\begin{align*}
		P_t^* = (1+\mu) P_{t}MC_{t}
	\end{align*}
	\item If $\theta>0$, then the optimal reset price is a markup over a weighted average of expected future marginal costs:
	\begin{align*}
		P_t^* &=  (1+\mu)E_t\left\{\sum_{s=0}^{\infty}\omega_{t,t+s}P_{t+s}MC_{t+s|t}\right\} \\
		\text{where }&\omega_{t,t+s}=\frac{\theta^s\Lambda_{t,t+s}Y_{t+s}P_{t+s}^{\epsilon-1}}{\sum_{k=0}^{\infty}\theta^k\Lambda_{t,t+k}Y_{t+k}P_{t+k}^{\epsilon-1}}
	\end{align*}
\end{itemize}
\end{frame}


\begin{frame}
\frametitle{Completing the Model
}
\begin{itemize}
	\item Because of CRS, real marginal cost is:
	\begin{align*}
		MC_{t+s|t}=\frac{W_{t+s}/P_{t+s}}{Y_{t+s|t}/N_{t+s|t}}=\frac{W_{t+s}/P_{t+s}}{A_{t+s}}=MC_{t+s}
	\end{align*}
	\item Aggregate output is:
	\begin{align*}
Y_t=\left[\int_0^1[A_tN_t(i)]^{\frac{\epsilon-1}{\epsilon}}di\right]^{\frac{\epsilon}{\epsilon-1}}=A_tN_t\left[\int_0^1\left(\frac{N_t(i)}{N_t}\right)^{\frac{\epsilon-1}{\epsilon}}di\right]^{\frac{\epsilon}{\epsilon-1}}
	\end{align*}
\begin{itemize}
	\item Term in brackets is loss in output due to misallocation caused by price dispersion.
	\item Creates welfare costs of inflation, but it is second order and drops out of log-linearization.
\end{itemize}
\end{itemize}
\end{frame}

\section{Next Steps}

\begin{frame}
\frametitle{Completing the Model
}
\begin{itemize}
	\item Combine price setting ``block'' with household ``block'' and monetary policy ``block'' to arrive at the new Keynesian model.
\end{itemize}
\end{frame}


\end{document}