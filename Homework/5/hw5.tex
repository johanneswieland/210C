

\documentclass{article}

\usepackage{amssymb}
\usepackage{graphicx}
\usepackage{amsmath}
\usepackage{amsfonts}
\usepackage[comma,authoryear]{natbib}
\usepackage{theorem}
\usepackage[onehalfspacing]{setspace}
\usepackage{indentfirst}
\usepackage{float}
\usepackage{geometry}
\usepackage{enumerate}
\usepackage{textcomp}


\usepackage{tikz}
\usetikzlibrary{intersections,calc}

\usepackage{mathabx}

\usepackage{url}

\setcounter{MaxMatrixCols}{10}

\newtheorem{theorem}{Theorem}
\newtheorem{acknowledgement}{Acknowledgement}
\newtheorem{algorithm}{Algorithm}
\newtheorem{axiom}{Assumption}
\newtheorem{case}{Case}
\newtheorem{claim}{Claim}
\newtheorem{conclusion}{Conclusion}
\newtheorem{condition}{Condition}
\newtheorem{conjecture}{Conjecture}
\newtheorem{corollary}{Corollary}
\newtheorem{criterion}{Criterion}
{\theorembodyfont{\rmfamily}
\newtheorem{definition}{Definition}
}
\newtheorem{example}{Example}
\newtheorem{exercise}{Exercise}
\newtheorem{lemma}{Lemma}
\newtheorem{notation}{Notation}
\newtheorem{problem}{Problem}
\newtheorem{proposition}{Proposition}
\newtheorem{remark}{Remark}
\newtheorem{solution}{Solution}
\newtheorem{summary}{Summary}
\newenvironment{proof}[1][Proof]{\noindent\textbf{#1.} }{\ \rule{0.5em}{0.5em}}
\geometry{left=1in,right=1in,top=1in,bottom=1in} 

\newcommand{\E}{\mathbb{E}}
\newcommand{\R}{\mathbb{R}}
\newcommand{\Z}{\mathbb{Z}}
\newcommand{\X}{\mathbb{X}}
\newcommand{\1}{\mathbf{1}}

\newcommand{\suchthat}{\;\ifnum\currentgrouptype=16 \middle\fi|\;}

\newcommand\invisiblesection[1]{%
  \refstepcounter{section}%
  \addcontentsline{toc}{section}{\protect\numberline{\thesection}#1}%
  \sectionmark{#1}}

\def\citeapos#1{\citeauthor{#1}'s (\citeyear{#1})}

\begin{document}

\title{Econ 210C Homework 5}
\author{Instructor: Johannes Wieland}
\date{\color{red} Due: no due date.}
\maketitle


%\section*{1. Sticky-Information Model}
%Gali, Exercise 3.6

\section*{1. Cost-push shocks}
Consider the standard new Keynesian model
\begin{align}
	\hat{y}_{t}&=E_{t}\hat{y}_{t+1} - E_{t}(\hat{i}_{t}-\hat{\pi}_{t+1}) \label{nk1} \\
	\hat{\pi}_{t}&=\beta E_{t}\hat{\pi}_{t+1} +\kappa (\hat{y}_{t}-\hat{y}_{t}^{eff}) + u_t \label{nk2} \\
	\hat{i}_{t}&=\phi_\pi \hat{\pi}_t,\qquad \phi_\pi>1 \label{nk3}
\end{align}
\begin{enumerate}[(a)]
\item Interpret each of the equations \eqref{nk1}-\eqref{nk3} (max 2 sentence each).
\item Assume $\hat{a}_{t}=0$ and $u_{t}=\rho_u u_{t-1}+\epsilon_t^u$ with $\epsilon_t^u\sim N(0,\sigma^2_{\epsilon^u})$. Solve for the equilibrium levels of $\hat{y}_t$, $\hat{y}_t-\hat{y}_t^{eff}$, $\hat{\pi}_t$, $\hat{i}_t$, and $\hat{r}_{t}=\hat{i}_t-E_t\hat{\pi}_{t+1}$ as a function of $u_t$.
\item Modify the Jupyter notebook ``newkeynesianlinear.ipynb'' to verify that your solution in (b) are correct.
\item Explain intuitively how a supply shock affects the output gap, inflation, the nominal interest rate, and the real interest rate. (4 sentences should suffice.)
\item Use your solution to express the loss function $L=\vartheta var(\hat{y}_{t}-\hat{y}_{t}^{eff})+var(\hat{\pi}_t)$ as a function of the model parameters, where $var(\hat{y}_{t}-\hat{y}_{t}^{eff})$ is the variance of the output gap and $var(\hat{\pi}_t)$ is the variance of inflation.
\item Show that the optimal interest rate rule satisfies $\phi_\pi=\rho_u + \frac{\kappa (1-\rho_u)}{\vartheta(1-\beta\rho_u)}$.
\item Using the optimal $\phi_\pi$, show that $\hat{y}_{t}-\hat{y}_{t}^{eff}=-\frac{\kappa}{\vartheta(1-\beta\rho_u)}\hat{\pi}_t$.
\item The optimal monetary policy under discretion is $\hat{y}_{t}-\hat{y}_{t}^{eff}=-\frac{\kappa}{\vartheta}\hat{\pi}_t$. Does the optimal $\phi_\pi$ deliver a better, a worse, or the same loss? Explain intuitively. (No derivation should be necessary.)
%\item Does the optimal $\phi_\pi$ deliver a better, a worse, or the same loss than optimal monetary policy under commitment ($\hat{x}_t-\hat{x}_{t-1}=\frac{\kappa}{\vartheta}\hat{\pi}_t$)? Explain intuitively. (No derivation should be necessary.)
\end{enumerate}

\section*{2. Estimating an Interest Rate Rule}
\begin{enumerate}[(a)]
	\item Download quarterly data from Fred for CPI inflation rate, the output gap, and the Federal Funds Rate from 1985Q1 to 2007Q4. Plot each data series.
	\item Estimate the Interest Rate Rule via OLS:
	\begin{align*}
		i_{t} = \alpha + \rho_i i_{t-1} + \phi_{\pi}\pi_{t} + \phi_{y}\tilde{y}_t + v_t
	\end{align*}
	and report your estimates.
	\item Explain why the OLS estimates are likely biased. 
	\item In which direction does the bias go?
%	\item (Optional) Describe a feasible procedure that addresses the identification problem and implement it. Contrast your estimates with the OLS estimates.
\end{enumerate}


\section*{3. Estimating an Interest Rate Rule in the NK model}
Implement the linearized new Keynesian model in a Jupyter notebook. 
\begin{align*}
	\tilde{y}_t &=-\sigma\left(\hat{i}_t-E_t\{\hat{\pi}_{t+1}\}\right)+E_t\{\tilde{y}_{t+1}\} \\
	\hat{\pi}_t&=\kappa \tilde{y}_{t} +\beta E_t \{\hat{\pi}_{t+1}\} + \hat{u}_t \\
	\hat{i}_t&= \phi_\pi\hat{\pi}_t + \bar{i}_t \\
	\bar{i}_t&=\rho_i \bar{i}_{t-1}+\epsilon^i_t \\
	\hat{u}_{t}&=\rho_u \hat{u}_{t-1}+\epsilon^u_t 
\end{align*}
Parameters: $\sigma=1,\kappa=0.03,\beta=0.99, \rho_i = 0.8,\rho_u=0.8$, and use the interest rate rule parameters you estimated in the previous question. Set the standard deviation of the monetary shock to 0.1 and that of the cost-push shock to 0.01.
\begin{enumerate}[(a)]
	\item Plot the IRFs. 
	\item Intuitively explain how the shocks $\epsilon^r_t$ and $\epsilon^i_t$ affect $\tilde{y}_t,\tilde{\pi}_t,\hat{i}_t$.
	\item Simulate a time series of length 1000 and plot it.
	\item Estimate the interest rate rule on your simulated data.
	\item Explain why this procedure does not recovery the true parameters.
	\item Would you be able to identify the true parameters using a Cholesky decomposition? Explain why or why not.
	\item What data do you need to identify the parameters of the interest rate rule?
	\item Implement your proposed approach and show that it correctly recovers $\phi_\pi$ in the model.
	\item Suppose the true interest rate rule was $\hat{i}_t= \phi_\pi\hat{\pi}_t + \phi_y\hat{y}_t + \epsilon^i_t$. Would you be able to identify both $\phi_\pi$ and $\phi_y$ using your procedure? Explain.
%	\item\label{sigma} Using your data from part 2, compute the standard deviation of the output gap and inflation over 1985Q1 to 2007Q4. In the model find the standard deviations of $\epsilon^r_t$ and $\epsilon^u_t$, that exactly replicate the standard deviations in the data. (Note: inflation in the model is not annualized---make sure your data counterpart is also not annualized.)
%	\item The loss function is
%	\begin{align*}
%		L_t = \frac{1}{2(1-\beta)}[var(\hat{\pi}_t)+\vartheta\ var(\tilde{y}_t)]
%	\end{align*}
%	Compute the loss under this loss function with $\vartheta=\kappa/\epsilon$ where the elasticity of substitution across varieties is $\epsilon=10$.
%	\item Replace the interest rate rule with the optimal discretionary policy. Compute and plot the IRFs. (You should keep the same $\sigma^r$ and $\sigma^u$ from part \ref{sigma}.)
%%	\item Intuitively explain how the shocks $\epsilon^r_t$ and $\epsilon^u_t$ affect $\tilde{y}_t,\tilde{\pi}_t,\hat{i}_t$ under the optimal discretionary policy.
%	\item Compute the loss under the optimal discretionary policy.
%	\item Replace the interest rate rule with the optimal commitment policy. Compute and plot the IRFs. (You should keep the same $\sigma^r$ and $\sigma^u$ from part \ref{sigma}.)
%%	\item Intuitively explain how the shocks $\epsilon^r_t$ and $\epsilon^u_t$ affect $\tilde{y}_t,\tilde{\pi}_t,\hat{i}_t$ under the optimal commitment policy.
%	\item Compute the loss under the optimal commitment policy.
%	\item Compare the losses under the interest rate rule, the optimal discretionary policy, and the optimal commitment policy. Interpret your findings.
\end{enumerate}

%\section*{2. Optimal Interest Rate Rule}
%Gali, Exercise 5.1








\end{document}
