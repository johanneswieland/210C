

\documentclass{article}

\usepackage{amssymb}
\usepackage{graphicx}
\usepackage{amsmath}
\usepackage{amsfonts}
\usepackage[comma,authoryear]{natbib}
\usepackage{theorem}
\usepackage[onehalfspacing]{setspace}
\usepackage{indentfirst}
\usepackage{float}
\usepackage{geometry}
\usepackage{enumerate}
\usepackage{textcomp}


\usepackage{tikz}
\usetikzlibrary{intersections,calc}

\usepackage{mathabx}

\usepackage{url}

\setcounter{MaxMatrixCols}{10}

\newtheorem{theorem}{Theorem}
\newtheorem{acknowledgement}{Acknowledgement}
\newtheorem{algorithm}{Algorithm}
\newtheorem{axiom}{Assumption}
\newtheorem{case}{Case}
\newtheorem{claim}{Claim}
\newtheorem{conclusion}{Conclusion}
\newtheorem{condition}{Condition}
\newtheorem{conjecture}{Conjecture}
\newtheorem{corollary}{Corollary}
\newtheorem{criterion}{Criterion}
{\theorembodyfont{\rmfamily}
\newtheorem{definition}{Definition}
}
\newtheorem{example}{Example}
\newtheorem{exercise}{Exercise}
\newtheorem{lemma}{Lemma}
\newtheorem{notation}{Notation}
\newtheorem{problem}{Problem}
\newtheorem{proposition}{Proposition}
\newtheorem{remark}{Remark}
\newtheorem{solution}{Solution}
\newtheorem{summary}{Summary}
\newenvironment{proof}[1][Proof]{\noindent\textbf{#1.} }{\ \rule{0.5em}{0.5em}}
\geometry{left=1in,right=1in,top=1in,bottom=1in} 

\newcommand{\E}{\mathbb{E}}
\newcommand{\R}{\mathbb{R}}
\newcommand{\Z}{\mathbb{Z}}
\newcommand{\X}{\mathbb{X}}
\newcommand{\1}{\mathbf{1}}

\newcommand{\suchthat}{\;\ifnum\currentgrouptype=16 \middle\fi|\;}

\newcommand\invisiblesection[1]{%
  \refstepcounter{section}%
  \addcontentsline{toc}{section}{\protect\numberline{\thesection}#1}%
  \sectionmark{#1}}

\def\citeapos#1{\citeauthor{#1}'s (\citeyear{#1})}

\begin{document}

\title{Econ 210C Homework 2}
\author{Instructor: Johannes Wieland}
\date{\color{red} Due: 5/13/2022, 11:59PM PST, on your Github repository.}
\maketitle




\section*{1. Complementarity of Money and Consumption}
Suppose the utility function in our classical monetary model is now
\begin{align*}
	U(X_t,L_t)=\frac{X_t^{1-\gamma}-1}{1-\gamma}-\chi \frac{N_t^{1+\varphi}}{{1+\varphi}}
\end{align*}
where $X_t$ is a composite of consumption and money,
\begin{align*}
	X_t=\left[(1-\vartheta)C_t^{1-\nu}+\vartheta\left(\frac{M_t}{P_t}\right)^{1-\nu}\right]^{\frac{1}{1-\nu}}
\end{align*}
\begin{enumerate}[(a)]
	\item Derive the first order conditions for this economy.
	\item Under what conditions does this economy predict that money is neutral? Explain why.
	\item Solve analytically for the steady state of the model (as far as you can), assuming $A=1$.
	\item Based on your steady state equations describe an algorithm for how to solve for the steady state.
	\item How would you calibrate $\vartheta$ given knowledge of $\nu$? (I.e., what moments of the data would you use and how?)
	\item Given knowledge of other parameters, how would you set $M$ such that $P=1$ in steady state?
	\item Derive the log-linearized model. 
	\item Following your calibration strategy for each of $\nu\in\{0.25,0.5,1,2,4\}$, solve the model using sequence space methods using the following parameters:
	\begin{align*} \gamma=1,\varphi=1,\chi=1,\beta=0.99,\rho_m=0.99
	\end{align*}
	where $m_t = \rho_m m_{t-1} + \epsilon_t^m$.
	
	Report the IRFs for consumption, prices, the nominal interest rate. Your graph for each variable should contain all five cases, appropriately labelled. 
	\item Intuitively explain your results.
	\item If you had evidence that an increase in the money supply increases consumption, which values for $\nu$ can you rule out? Explain why.
	\item Make sure your code packet contains a file that produces your graphs with a single click. (It does not need to save the graphs.) Upload it to Github.
\end{enumerate}


%\section*{2. Controlling Inflation}
%
%The Fisher equation is
%	\begin{align*}
%		i_t = r_t + \E_t(\pi_{t+1}) \\
%	\end{align*}
%
%The money demand equation is
%\begin{align*}
%	m_t - p_t &= c_t - \frac{1}{\nu}i_t + u_d^d - u_t^s
%\end{align*}
%
%All exogenous variables AR(1) processes:
%\begin{align*}
%	c_t &= \rho_c c_{t-1} + \eta_t^c \\
%	r_t &= \rho_r r_{t-1} + \eta_t^r \\
%	u_t^d &= \rho_d u^d_{t-1} + \eta_t^d \\
%	u^s_t &= \rho_s u^s_{t-1} + \eta_t^s 
%\end{align*}
%
%\begin{enumerate}[(a)]
%	\item Suppose that the central bank sets interest rates according to an interest rate (Taylor) rule:
%	\begin{align*}
%		i_t = \bar{i}_t + \phi \pi_t, \qquad \phi>1
%	\end{align*}
%	Solve for the equilibrium rate of inflation, assuming the terminal condition $\lim_{T\rightarrow\infty}\frac{1}{\phi^{T}}\pi_{t+T}=0$ holds.
%	\item\label{optpol} The central bank has a sequence of inflation targets $\{\pi_{t+s}^*\}_{s=0}^{\infty}$. Define the control error as
%	\begin{align*}
%		\epsilon_t = \pi_t - \pi_t^*
%	\end{align*}
%	Derive an expression for how the central bank should set $\bar{i}_t$ to set the control error to zero.
%	\item Describe this policy in words.
%	\item The Fisher equation reflects expectations of the private sector. Suppose the central bank had different expectations $\E_t^{CB}$ based on private information. How would this affect your answer to part (\ref{optpol})?
%	\item Suppose the central bank now sets the money supply $m_t$ rather than the nominal interest rate $i_t$. Derive the rate of inflation for this case.
%	\item Derive an expression for how the central bank should set $m_t$ to set the control error to zero.
%	\item In words, why might a central bank prefer to set policy based on $i_t$ rather than $m_t$ and vice-versa?
%\end{enumerate}



\end{document}
